% Chapter Template
\documentclass{report}
\begin{document}
\chapter{Intoduction} % Main chapter title

%\label{Chaptery 1} % Change X to a consecutive number; for referencing this chapter elsewhere, use \ref{ChapterX}

%\lhead{Chapter 1. \emph{Sample}} % Change X to a consecutive number; this is for the header on each page - perhaps a shortened title

%----------------------------------------------------------------------------------------
%	SECTION 1
%---------------------------------------------------------------------------------------
%\section{}





The advent of credit cards and their increasing functionality have not only given people more personal comfort, but have also attracted malicious characters interested in the handsome rewards to be earned. Credit cards are nice target for fraud, since in a very short time a lot of money can be earned without taking many risks. This is because often the crime is only discovered a few weeks after the date. Credit card fraud can be defined as “unauthorized account activity by a person for which the account was not intended. Operationally, this is an event for which action can be taken to stop the abuse in progress and incorporate risk management practices to protect against similar actions in the future”. In simple terms, credit card fraud is defined when an individual uses another individual’s credit card for personal benefit while the owner of the card and the card issuer are not aware of the fact that the card is being misused. And the person using the card has not at all having the connection with the card holder or the issuer and has no intention of making the repayments for the purchase they done. 
%\subsection{•}


%\section{Adding another section}
%You can show a lot of figures together like these 


%Tables can be added like this

\chapter{DIFFERENT TYPES OF FRAUD TECHNIQUES
}

There are three classes of frauds namely card related, merchant related and internet frauds. Some of them are listed below
\paragraph{}

\section{Card Related Frauds}
\paragraph{}

\subsection{Lost/Stolen Card:}
 This type of fraud occurs when the fraudster simply steals a customer’s card. In this case, the customer might feel he has lost his card, but actually this card might have been acquired by an attacker. 

\subsection{Account Takeover:}
This type of fraud occurs when the valid customer’s personal information is taken by fraudsters. The fraudsters takes control of a legitimate account by either providing the customer’s account number or the card number. The fraudster then contacts the card issuer, as the genuine card holder, to ask the mail to redirect to a new address. The fraudster reports card lost and asks for a replacement to be sent. 

\subsection{Cardholder-Not-Present (CNP):}
CNP transactions are performed only on the internet that is remotely, in such kind of frauds neither the card nor the cardholder is present at the point-of-sale. This take many types of transactions such as orders made over the phone or Internet, by mail order or fax. In such transactions, retailers are unable to physically check user or identity of the card holder which makes the user unknown and able to disguise their true identity, The details of the credit card are normally copied without the cardholder’s knowledge, collected from the receipts thrown by the customer or obtained by skimming process. Frequently obtained card details are generally used with fabricated personal details to make fraudulent CNP purchases. This means that while the three or four digit card security code (CVV number) on the back of cards can help prevent fraud where card details have been obtained, but when the card is stolen it won’t be helpful.

\subsection{Fake and Counterfeit Cards:}
This is another type of fraud where the creation of the counterfeit cards, together with lost or stolen cards poses highest threat in credit card frauds. Fraudsters are constantly finding new and more innovative ways to create counterfeit cards. The below mentioned are some of the techniques used for creating false and counterfeit cards.  

\subsection{Erasing the Magnetic strip:}
This is the type of fraud where the fraudsters erase the magnetic stripe by using the powerful electromagnet. The fraudsters then tampers with the details on the card so that they match the details of a valid card, which they may have attained, for example, when the fraudster begins to use the card, the cashier will swipe the card through the terminal several times, before realizing that the metallic strip does not work. The cashier will then proceed to manually input the card details into the terminal. This kind of fraud is having high risk because the cashier will be looking at the card closely to read the numbers.

\subsection{Phishing:}
Phishing is a type of fraud designed to steal a person’s identity. It is usually committed via spam e-mail or pop-up windows. Phishing works by a malicious person sending lots of false emails. The emails looks like they have come from a website or company you trust, for example your bank. The message tells you to provide the company with your personal details including your payment card details. They can claim that the reason for this is a database crash or something like this. To make the email look even more authentic, the fraudster might put a link to a website that look exactly like the real one but in fact that is not the real one and that is the fake one. These copies are often called “Spoofed websites”. When you are on the spoofed site they can ask you for even more personal details that will be directly transmitted to the person who made that website.  

\section{Merchant Related Frauds}
\paragraph{}
\subsection{Merchant Collusion:}
This type is done when a merchant purposely passes on his customer’s personal information to fraudsters. 

\subsection{Triangulation:}
Here, the fraudster creates a fake website and operates from there. Many discounts are given to the customers through this website due to which users are attracted to such websites. They purchase items and there they enter their personal information. Then this information is obtained by the fraudsters and they use it to perform illegimitate transactions.  

\paragraph{}
\section{Internet Frauds}
\paragraph{}
\subsection{False Merchant Sites:}
 In this type, the website asks the customers to enter their personal details if they want to access the content of the website. In this way, these fraudsters collect many credit card number which they use later for performing fraudulent transactions.

\subsection{Keystroke Loggers:}
“Keystroke logger” is a spyware which infects a user’s computer unknown to him. This spyware tracks all the details typed by the user and gives this information to the fraudster who thus obtains all the personal details.

\subsection{Cell phone camera Scan:}
When a customer is paying his bills, a fraudster may be roaming somewhere near him. The customer may be under the assumption that the attacker is busy chatting on his phone, but actually he is taking digital image of the computer’s details such as card number, expiry date, etc. This type of fraud is possible because of powerful cameras used these days. 

\subsection{Site Cloning:}
Site cloning is where fraudsters close an entire site or just the pages from which the customer made a purchase. Customers have no reason to believe they are not dealing with the company that they wished to purchase goods or services from because the pages that they are viewing are identical to those of the real site. The cloned site will receive these details and send the customer a receipt of the transaction through the email just as the real company would do. The customer suspects nothing, while the fraudsters have all the details they need to commit credit card fraud.

\chapter{PROBLEMS WITH CREDIT CARD FRAUD DETECTION}

One of the biggest problems associated with fraud detection is the lack of both literature providing experimental results and of real world data for academic researchers to perform experiments on. This is because fraud detection is often associated with sensitive financial data that is kept confidential for reasons of customer privacy. 
\paragraph{}
We now enumerate some of the properties a fraud detection system should have in order to perform good results. 

\begin{itemize}
  \item The system should be able to handle skewed distributions, since only a very small percentage of all credit card transactions is fraudulent, To solve this problem, often the training sets are divided into pieces where the distribution is less skewed.

  \item The ability to handle noise. This is simply the presence of errors in the data, for instance incorrect dates. Noise in actual data limits the accuracy of generalization that can be achieved, no matter how extensive the training set is. One way to deal with this problem is by cleaning the data. 
  
  \item Overlapping data is another problem in this field. Many transactions may resemble fraudulent transactions, when actually they are legitimate. The opposite also happens, when a fraudulent transaction appears to be normal. 


\item The system should be able to adapt themselves to new kinds of fraud. Since after a while successful fraud techniques decrease in efficiency, due to the fact that they become well known. Then a “Good” fraud tries to find new and inventive ways of doing his job.

\item There is a need for good metrics to evaluate the classifier system. As an example, the overall accuracy is not suited for evaluation on a skewed distribution, since even with a very high accuracy, almost all fraudulent transactions can be misclassified. 

\item The system should take into account the cost of the fraudulent behaviour detected and the cost associated with stopping it. For example, no profit is made by stopping a fraudulent transaction of only a few Euros. 


\end{itemize}

\end{document}

